\newpage
\section{\textbf{Appendix A. Follow-up Interview Transcript}}

Transcript of the follow-up interview: Interviewer (\textbf{Yernur:}), Interviewee (\textbf{\textit{Adel:}})

Time of the interview: 13:40 on Sunday

\textbf{Yernur:}

Привет.

Hello.

\textbf{\textit{Adel:}}

Привет.

Hello.

\textbf{Yernur:}

Как дела?

How are you?

\textbf{\textit{Adel:}}

Хорошо.

Good

\textbf{Yernur:}

Отлично, супер! Ну давай начнем тогда интервью. Ты согласна?

Great, cool! Now let’s start the interview. Do you agree?

\textbf{\textit{Adel:}}

Да.

Yes

\textbf{Yernur:}

Отлично. И первый вопрос будет таким. Предпочла бы, чтобы переход на расписание с началом 08:00 перешел во время летних каникул, а не во время четверти, как это реализовала наша администрация?

Excellent. And the first question will be like this. Would you prefer to switch to the schedule with the beginning of 08:00 during the summer holidays, and not during the quarter, as our administration implemented it?

\textbf{\textit{Adel:}}

Ну, честно говоря, да. Потому что перестроиться прямо во время четверти было очень тяжело. Во первых, потому что уже был другой режим сна, и режим дня, еще потому что очень многие, так же я уже начал какие то курсы, возможно, подготовки к IELTS-у или вообще другие хобби вне школы, чтобы наработать портфолио. И вот когда произошла эта смена расписания, было очень тяжело обратно, все эти расписания подстроить под школу.

Well, to be honest, yes. Because it was very difficult to readjust right during the term. Firstly, because there was already a different sleep mode, and a day mode, also because a lot of people, I also already started some courses, perhaps preparing for IELTS, or even other hobbies outside of school in order to develop a portfolio. And when this schedule change took place, it was very difficult to adjust all these schedules back to school.

\textbf{Yernur:}

Ясно. То есть вот ты сказала о том, что были сформированы режим сна, то есть во время каникул режим сна меняется.

I see. That is, you said that a sleep schedule was formed, that is, during the holidays, the sleep regime changes.

\textbf{\textit{Adel:}}

Ну, как бы в промежуток каникул легче перестроить свой режим сна под новое  расписание, нежели когда ты просыпался в восемь, и засыпал в 12, и был полный сон с переходом на то, что тебе нужно ложиться примерно также, но просыпаться уже в 06:00. Ну, то есть получается недосып. И от этого усталость и от этого ещё тяжелее сфокусироваться на уроках.

Well, during the holidays it is easier to adjust your sleep schedule to a new schedule than when you woke up at eight and fell asleep at 12, and had a full sleep with the transition to the fact that you need to go to bed about the same, but wake up at 06:00. Well, that is, it turns out lack of sleep. And from this fatigue and from this it is even harder to focus on lessons.

\textbf{Yernur:}

Ясно, эм далее. Вот ты сказала то, что у нас слишком много 20-минутных перемен в прошлом интервью. И я так подумал, что, возможно, это связано с тем, что, возможно, удобнее подавать еду, когда такой формат 20 минутных перемен, а не когда обед идет во время того, пока другой класс учится на уроках, полностью. Получается, насколько изменилась подача еды в столовой после перехода на новое расписание? Если вообще изменилась?

Okay, um next. Here you said that we have too many 20-minute changes in the last interview. And I thought that maybe this is due to the fact that it may be more convenient to serve food when such a format is 20-minute breaks, and not when lunch is going on while another class is studying in class, completely. It turns out, how much has the serving of food in the dining room changed after switching to a new schedule? If it has changed at all?

\textbf{\textit{Adel:}}

Ну, вообще она как бы ускорилась. Потому что раньше нам приходилось вот как начинается перемена 40 минут, или как там было раньше. Мы ждали еще минут 10-15, пока нам накроют стол и только запустят столовую. И, получается, уже проходит минут 20-30 пока ты пообедаешь. А сейчас мы идем на завтрак или на обед. И еда уже готова и не приходится тратить время. Но насчет самой качества еды я не могу сказать, потому что сама не ем в школе.

Well, in general, it kind of accelerated. Because before we had to do this is how the break starts for 40 minutes, or whatever it was before. We waited another 10-15 minutes until the table was set for us and the dining room was just launched. And, it turns out, 20-30 minutes have already passed while you have lunch. And now we're going to breakfast or lunch. And the food is already ready and you don't have to waste time. But I can't say about the quality of the food itself, because I don't eat at school myself.

\textbf{Yernur:}

А ясно хорошо. И тогда последний вопрос. А в прошлом интервью ты говорила том, что большое количество уроков в день негативно влияет на твою продуктивность. И как бы ты отреагировала на уменьшение количества уроков в день? Но при этом введения обучения в субботу.

Okay, good. And then the last question. And in the last interview, you said that a large number of lessons per day negatively affects your productivity. And how would you react to a reduction in the number of lessons per day? But at the same time the introduction of training on Saturday.

\textbf{\textit{Adel:}}

Скорее негативно, чем позитивно. Потому что, ну, суббота и воскресенье они как выходные идут, чтобы отдохнуть можно было. И также у меня уже стоят другие курсы и другие планы. И было бы очень тяжело обратно вернуться и обратно вернуться на 6-дневное обучение. Поэтому уже легче было бы все 5 дней отучиться и потом отдыхать два дня.

More negative than positive. Because, well, Saturday and Sunday are like weekends, so that you can relax. And also I already have other courses and other plans. And it would be very difficult to go back and go back to 6-day training. Therefore, it would be easier to study all 5 days and then rest for two days.

\textbf{Yernur:}

Ясно. То есть ты вообще за уменьшение количества уроков, но при этом увеличение самостоятельной работы.

I get that. That is, you are generally in favor of reducing the number of lessons, but at the same time increasing independent work.

\textbf{\textit{Adel:}}

Ну, я думаю, так даже удобнее для некоторых учеников, потому что и информация легче усваивается и лучше усваивать.

Well, I think it's even more convenient for some students, because information is easier to digest and better to assimilate.

\textbf{Yernur:}

Хорошо, спасибо. На этом наше интервью закончено. Пока.

Well, thank you. This concludes our interview. Bye.

\textbf{\textit{Adel:}}

Пока

See you