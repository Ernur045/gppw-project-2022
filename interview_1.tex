\newpage
\section{\textbf{Appendix A. Main Interview Transcript}}

Topic: \textbf{How Does School Schedule Affect Eleventh Grade Students in Nazarbayev Intellectual School of Physics and Mathematics in Nur-Sultan?
Transcript of the interview: Interviewer (\textbf{Yernur}), Interviewee (Adel)}
Time of the interview: 20:00 on Sunday

\textbf{Yernur}: 

Добрый день, сейчас мы начнем интервью, и начнем с вопроса ты сова или жаворонок?

Good afternoon, now we will start the interview, and we will start with the question are you an owl or an early riser?

Adel:

Я лично жаворонок

I am personally an early riser

\textbf{Yernur}:

Хорошо, и во сколько ты просыпаешься в школьные дни?

Okay, and what time do you wake up on school days?

Adel:

Обычно, в 6 утра

Usually, at 6 am

\textbf{Yernur}: 

В 6 утра, хорошо, и насколько сложно или просто?

At 6 am, ok, and how difficult or simple is it?

\textbf{\textit{Adel}}

Это довольно тяжело, потому что обычно я ложусь в 12 ночи, и просыпаюсь в 6. И поэтому 6 часов сна не хватает

It's pretty hard because I usually go to bed at 12 at night and wake up at 6. And that's why 6 hours of sleep is not enough

\textbf{Yernur}:

Насколько тебе сложно просыпаться в выходные?

How hard is it for you to wake up on the weekend?

\textbf{\textit{Adel}}

Ааа, бывает тяжело, потому что нету и необязательно вставать очень рано, но я стараюсь вставать хотя бы до 10 

Aah, it can be hard, because there is no need to get up very early, but I try to get up at least before 10

\textbf{Yernur}: 

Ясно, давай перейдём на твоё самочувствие в будние дни. Как ты себя чувствуешь на первом уроке? (Может сонной, не можешь сконцентрироваться, или энергичной, или может быть даже головные боли есть?)

Okay, let's move on to how you feel on weekdays. How do you feel in the first lesson? (Maybe sleepy, can't concentrate, or energetic, or maybe even have headaches?)

Adel:

Ну на первых уроках я в основном более сконцентрированная и могу как-бы… голова работает лучше, и мне кажется на первых уроках

Well, in the first lessons I'm mostly more concentrated and I can sort of... my head works better, and it seems to me in the first lessons

\textbf{Yernur}: 

Ясно, а на уроках которые идут после первого и до обеда.

Okay, and on the lessons that go after the first and before lunch.

Adel:

Ну в принципе так же, но ближе к обеду уже начинаешь уставать от количества уроков и нагрузки, поэтому там уже больше усталость 

Well, basically the same, but closer to lunch you already start to get tired of the number of lessons and the load, so there is more fatigue

\textbf{Yernur}: 

Ясно, а в послеобеденные уроки?

Okay, but in the afternoon lessons?

\textbf{\textit{Adel}}

На послеобеденных уроках уже тяжелее, потому что уже проходит 10 уроков и очень сильно тяжело сосредоточиться на уроке, на учителе и в принципе делать какую-то работу

It's harder in the afternoon lessons, because 10 lessons are already taking place and it's very hard to focus on the lesson, on the teacher and, in principle, to do some work

\textbf{Yernur}: 

Ага, хорошо, и как ты себя чувствуешь себя после школы, какая у тебя продуктивность после школы при выполнении домашнего задания?

Yeah, well, and how do you feel after school, what is your productivity after school when doing homework?

\textbf{\textit{Adel}}

В основном после школы я не могу нормально делать домашнее задание, потому что после 10 или обычно 8 уроков, я уже сильно устаю и предпочитаю поспать и делать это ближе к 12, либо делать рано утром, встать пораньше чем в 6 утра

Basically, after school, I can't do my homework properly, because after 10 or usually 8 lessons, I'm already very tired and prefer to sleep and do it closer to 12, or do it early in the morning, get up earlier than 6 in the morning

\textbf{Yernur}: 

Хорошо, понятно. Давай теперь поговорим о расписании в целом. Насколько сложно было перейти с расписания, которое ввели из-за эпидемиологической ситуации, то есть когда мы приходили в 9 утра, на нормальное расписание, когда мы начали приходить в 8 утра?

Okay, I see. Now let's talk about the schedule in general. How difficult was it to switch from the schedule that was introduced due to the epidemiological situation, that is, when we came at 9 a.m., to the normal schedule when we started coming at 8 a.m.?

Adel:

Аа, ну было тяжело, потому что как-бы, когда уроки начинались в 9 утра, у меня был один распорядок дня и у меня было больше времени на утренние процедуры, и была возможность делать уроки с утра. А, потом когда уроки переставили на 8 утра, пришлось вставать ещё раньше, но также у нас увеличилось количество времени, которое мы проводим в школе, поэтому и нагрузка прибавилась, поэтому я ложилась так же и вставала ещё раньше

Ah, well, it was hard, because, as it were, when lessons started at 9 in the morning, I had one daily routine and I had more time for morning procedures, and I had the opportunity to do lessons in the morning. And then, when the lessons were moved to 8 a.m., I had to get up even earlier, but we also increased the amount of time we spend at school, so the load increased, so I went to bed the same way and got up even earlier

\textbf{Yernur}:

Ага, и получается, какое расписание ты предпочитаешь? Первое или второе?

Yeah, and, which schedule do you prefer? The first or the second?

\textbf{\textit{Adel}}

Лично я предпочитаю первое

Personally, I prefer the fisrt

\textbf{Yernur}: 

Ага, и почему?

Yeah, and why?

\textbf{\textit{Adel}}

Потому что, ну во первых есть больше времени с утра, и необязательно вставать очень рано, и есть время чтобы выспаться, и набраться своих сил [inaudiable]. А также в новом расписании есть очень много больших перемен, которые по моему мнению занимают слишком много времени, которое можно было бы потратить на школьные занятия или тот же самые отдых?

Because, well, first of all, there is more time in the morning, and it is not necessary to get up very early, and there is time to sleep and gain your strength [inaudible]. And also there are a lot of big changes in the new schedule, which in my opinion take too much time, which could be spent on school classes or the same rest?

\textbf{Yernur}:

Получается, когда было ощущение большей продуктивности?

Wthere was a feeling of greater productivity?

\textbf{\textit{Adel}}

Эм, в старом расписании, когда начиналось в 9

Um, in the old schedule, when it started at 9

\textbf{Yernur}: 

Кроме того, что уроки начали начинаться в 8 утра, у нас добавились 4 дополнительные перемены по 20 минут. Как они влияют на твою продуктивность и концентрацию?

In addition to the fact that lessons started starting at 8 am, we have added 4 additional breaks of 20 minutes each. How do they affect your productivity and concentration?

\textbf{\textit{Adel}}

Я бы сказала негативно, средне то есть, а потому что, наверное в первой половине дня 20-минутные перемены, будут время немножко отдохнуть, перевести дух, и правда помогают. Однако уже после обеда одна перемена 20-минутная, о которой… одна или две, на которых честно говоря уже не знаешь что делать. И [inaudiable] недосыпаешь, либо уже теряешь концентрацию на уроках, и если вообще перемена стоит между двумя уроками, между одним уроком математики и вторым уроком математики, то потом уже тяжело вернуться в процесс урока и понять.

I would say negatively… I mean neutrally, but because, probably, in the first half of the day there is one 20-minute brake, there will be time to rest a little, take a breath, and they really help. However, after lunch there is one 20-minute break, about which... one or two, on which, to be honest, you no longer know what to do. And [inaudible] you don't get enough sleep, or you lose concentration in lessons, and if there is a change at all between two lessons, between one math lesson and the second math lesson, then it's hard to go back to the lesson process and understand.

\textbf{Yernur}: 

Интересно, так, и… сейчас секунду, как по вашему, какие ещё факторы, кроме этих 20-минутных перемен и начала уроков в 8 влияют на продуктивность учеников и лично на твою продуктивность.

It's interesting, so, and... one second, in your opinion, what other factors besides these 20-minute changes and the beginning of lessons at 8 affect the productivity of students and your productivity personally.

\textbf{\textit{Adel}}

Наверное, я бы сказала количество уроков в день, потому что слишком большое количество уроков [разрыв связи] 

I would probably say the number of lessons per day, because there are too many lessons [disconnection]

\textbf{Yernur}: 

Извиняюсь, пока ты отвечала, связь немного прервалась, можешь пожалуйста ещё раз повторить?

I'm sorry, while you were answering, the connection was interrupted a little, can you please repeat it again?

\textbf{\textit{Adel}}

Да-да, хорошо. Нормально слышно?

Yes, okay. Do you hear me?

\textbf{Yernur}: 

Да

Yes

Adel:

Хорошо. Ну я считаю что это количество предметов, уроков в день. Потому что бывает что в один день 10 уроков, а в другой день 6 уроков, и они могут все быть очень разными, что очень тяжело переключаться между ними, и запоминать материал по каждому из них. И наверное, количество материала, которое нужно изучать самостоятельно, потому что не всё можно пройти на уроке, большая часть может оставаться на самостоятельное изучение. И большое количество тоже тяжело доносить

Good. Well, I think this is the number of subjects, lessons per day. Because it happens that one day there are 10 lessons, and another day there are 6 lessons, and they can all be very different, which makes it very difficult to switch between them and memorize the material for each of them. And probably, the amount of material that needs to be studied independently, because not everything can be passed in the lesson, most of it can remain for independent study. And a large number is also hard to convey 

\textbf{Yernur}: 

Хм, ясно. И получается, с чем связано то, что немалая часть материала проходится самостоятельно дома? С тем что не хватает уроков, с тем что учителя что-то не так делают или что-то ещё? 

Hmm, I see. And son, what is the reason that a considerable part of the material is passed independently at home? With the fact that there are not enough lessons, with the fact that teachers are doing something wrong or something else?

\textbf{\textit{Adel}}

Наверное это от количества материала рассчитанного на одну четверть, потому что его может быть много и на уроках его могут объяснить в общем виде, кратко. А для полного понимания, для сдачи МЭСК-ов и в принципе понятия темы хорошо было бы изучать его самостоятельно дома. Поэтому большая часть остаётся на самостоятельное изучение, если это ученику интересно и нужно.

Probably it depends on the amount of material planned for one term, because there can be a lot of it and it can be explained in general, briefly in the lessons. And for a full understanding, for the delivery of the exams and, in principle, the concept of the topic, it would be good to study it yourself at home. Therefore, most of it remains for independent study, if it is interesting and necessary for the student.

\textbf{Yernur}: 

Ясно, спасибо. И тогда, Учитывая всё вышесказанное, как бы ты изменила расписание звонков и уроков? 

Okay, thanks. And then, given all of the above, how would you change the schedule of calls and lessons?

\textbf{\textit{Adel}}

Эм, ну как я говорила раньше мне в принципе устраивало расписание, которое было … но можно было бы оставить прежнее и убрать 20-минутные перемены, которых много, не считая тех, что на обед и на завтрак. Чтобы дать возможность ученикам поехать домой пораньше.

Um, well, as I said before, I was basically satisfied with the schedule that was ... but it would be possible to leave the same and remove the 20-minute changes, of which there are many, not counting those for lunch and breakfast. To give students the opportunity to go home early.

\textbf{Yernur}: 

Я снова извиняюсь, там снова связь пропала, пока ты отвечала. Можешь пожалуйста ещё раз?

I'm sorry again, the connection was lost again while you were answering. Can you please say it again?

\textbf{\textit{Adel}}

Да. Меня с какого момента не было слышно? 

Yes. I haven't been heard from since when?

\textbf{Yernur}: 

А, ну вот ты сказала где-то пару слов. Потом связь прервалась где-то секунд 5. 

Ah, well, you said a couple of words somewhere. Then the connection was interrupted for about 5 seconds.

\textbf{\textit{Adel}}

Хорошо. Расписание, которое начиналось в 9 утра было хорошим. Оно в меру было нагруженым. И было время чтобы отдохнуть. А расписание которое начинается в 8 утра, я бы убрала 20-минутные перемены. Чтобы у учеников была возможность вернуться домой пораньше и было больше свободного времени. 

Okay. The schedule that started at 9 a.m. was good. It was moderately loaded. And there was time to rest. And the schedule that starts at 8 a.m., I would remove the 20-minute changes. So that students have the opportunity to return home earlier and have more free time.

\textbf{Yernur}: 

Ясно, спасибо большое за твои ответы. На этом наше интервью закончено. До встречи

Okay, thanks a lot for your answers. This concludes our interview. See you soon

\textbf{\textit{Adel}}

До встречи.

See you soon